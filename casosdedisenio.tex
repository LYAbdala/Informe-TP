\subsection{Diseño Orientado a Objetos y justificación}

Para el diseño vamos a usar los siguientes escenarios posibles en el marco del uso de la aplicación:

\begin{enumerate}
\item Fernil fue a un bar llamado Los Trillizos, ubicado en microcentro. Él ya conocía el bar por lo cual no lo busco previamente en la app. Después de pasar un buen rato decide realizar un comentario sobre el bar para que más gente pueda apreciar lo qué es un buen bar.

\item Javier sale con un grupo de amigos,se encuentran cercanos a una avenida importante y cerca de un bar que no conoce. Para no pasar una mala noche, decide buscar en la app los bares cercanos a su posición. Pero entre los resultados no aparece el bar que esta viendo con sus propios ojos. Por otro lado los otros bares que le aparecen son bares que no le quedan lo suficientemente cerca. Dado esto decide entrar al bar. Finalmente, como pasó una muy buena noche y recordó que ese bar no estaba en la app, decide agregarlo a la app, Para ellos completa un par de datos como: el nombre del bar la dirección y una pequeña reseña sobre el mismo. 

\item Leila se encuentra en su casa, en espera de un grupo de amigos para terminar con un tp. Tiene dos opciones quedarse en casa o ir a un lugar cercano. Como no le gusta mucho la idea de que se queden en casa, decide buscar algo cerca. Usa la app con la intencción de buscar un bar cercano de su casa y que además tengan buenas puntuaciones con respecto a wifi y enchufez ya que van a usar de manera exaustiva sus respectivas laptos.

\item Dos amigos se encuentran en las cercanias de Plaza Italia, ambos vienen de lugares totalmente distintos pero deciden verse ahí porque cada uno tiene un colectivo de distancia. Una vez que se encuetran, no tienen ni idea a donde ir. Por medio de la app uno de ellos decide buscar algo cercano. Luego de ver un conjunto de bares cercanos se deciden por uno. Dado a que ninguno es de la zona prefieren que la app les muestre por pantalla una forma de dirigirse al bar que eligieron ir.

\item Después de un largo día de estudio, Leila y sus amigos por fin terminaron. Dentro del grupo de amigos se encuentra Andres, a él no le parecio que la cantidad de enchufez fueran las suficientes, debió se por el hecho que fue uno de los dos que se quedaron sin batería y no pudieron continuar trabajando con sus propias laptos. Dado esto Andres decide, y realmente enojado, realizar una votación de 0 para la cantidad de enchufez.  

\end{enumerate}


\subsubsection{Asunciones sobre el dominio del problema}

Para poder pensar en una forma más especifica, y sobre todo, para lograr alcanzar las metas que se nos fueron impuestas para los sprints. Fue decisión del grupo tomar las siguientes asunciones:

\begin{itemize}
\item Cualquier persona que quiera usar la app puede hacerlo, para cualquiera de sus funcionalidades. No tomamos en cuenta niveles de usuarios como: usuario premium, usuario básico, usuario registrado, usuario sin registrar.
\end{itemize}
