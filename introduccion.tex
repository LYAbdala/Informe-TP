Dada la pachorra de cierto grupo de estudiantes para juntarse a estudiar en una biblioteca, nos encontramos con el problema de localizar bares con ciertas caracteristicas del 
agrado de los pachorrientos estudiantes. Estos mismos proveyeron una muy breve y ambigua descripcion de su problema, el cual pretendemos resolver mediante el desarrrollo de una 
aplicacion web llamada Wi-FindBar (porque algo solo es cheto si esta en ingles). 

El presente informe muestra el diseno de la aplicacion, asi como el desarrollo de la misma hasta el dia de la fecha. Para el diseno se uso programacion orientada a objetos, 
principalmente las tecnicas de doble dispatch y polimorfismo. Para el desarrollo usamos la porqueria (...) de Scrum, ya que nos permite empezar a trabajar de manera incremental, sin tener
pleno conocimiento del cominio del problema.