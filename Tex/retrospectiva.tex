En esta seccion chamuyaremos sobre los problemas hallados y demas bobadas (...) ocurridas durante los sprint.

\subsection{Primer diseño}

En principio planteamos un diseño sobre el cual iteramos en cada sprint. A continuacion los describiremos brevemente para darl al lector una idea de la evolucion del diseño, sin embargo
no haremos demaciado incapie en esto pues no es el diseño final sobre el que trabajamos.

En nuestro diseño consideramos como centrales los siguientes puntos. 

Metodo buscar bares: La idea de este metodo era tener un mensaje polimorfico al que pudiesemos pasarle cualquier tipo de
requerimientos (los cuales probablemente deberian haber sido implementador mediante cloures) y devolviera la lista de bares que los cumpliesen. Para esto El objeto encargado 
de buscar los bares deberia conocer la forma de almacenarlos.

Roles: Nuestros principales roles se basaron en principio en los distintos tipos de permisos que tendrian los usuarios, siendo ellos No registrado, Registrado y Moderador. A cada 
uno de estos les correspondian diferentes funcionalidades que podian o no pertenecer a uno de los otros usarios. Por ejemplo, un usuario no registrado puede buscar al igual que uno registrado
o que un moderados, sin embargo, el moderador podia eliminar bares, cosa que un usuario registrado no podria.

Objeto bases de datos: Consideramos tres bases de datos diferentes, una para regitras usuarios, una para almacenar la informacion de los bares y otra par aalmacenar los comentarios y calificaciones
realizadas por los usuarios. Esta tres bases de datos estaban fuertemente acopladas entre si, ya que, por ejemplo, la base de datos de votaciones conocia a los bares y a los usuarios.

Objeto bar: El objeto bar contenia toda la informacion asociada al mismo. Es decir, nombre, direccion, pocision, sus calificaciones, comentarios y una brebe descripcion del mismo. Es 
decir, que el objeto bar contenia toda la informacion necesaria para responder por si mismo la toda la funcionalidad pedida. 

Metodos de administracion de bares: Estos metodos se referian a la creacion, eliminacion y modificacion de bares, teniendo acceso a la base de datos de bares y de votaciones.


Metodos de administracion de usuarios: Estos metodos se referian a la creacion, eliminacion y modificacion de usuarios, teniendo acceso a la base de datos de usuarios y de votaciones.


\subsection{Primer Sprint}

Durante el primer sprint detectamos problemas con el diseno de los usuarios. Al querer modificar el status de un usuario debiamos destruir el usuario y crearlo de nuevo, asignandole su nuevo
permiso. Esto fue debido a que habiamos definido una clase usuario abstracta de la cual se heredaban los usuarios logueado y no logueado. Para evitar esto, definimos un ingresante a la aplicacion sin ningun permiso, que solo podia buscar. Al loquearse este era destruido y se creaba
un nuevo objeto que representaba un usuario ya logueado.

Los nuevos roles que consideramos fueron seleccionados de acuerdo a la funcionalidad que proveia cada uno. Asi indentificamos a los roles Buscador de bares, Editor de guias, Comentador
de bares, Calificador de bares, descriptos en la seccion 2. A partir de estos surgieron nuevos objetos y metodos, y se modificaron los ya existentes.

Objeto Bar: Dada el alto acloplamiento que tenia este objeto, lo redisenamos intentando capturar solo su esencia. Esto resulto en que el objeto bar solo conoce su nombre y direccion. 
Los demas datos que almacenaba delegados a nuevos objetos.

Objeto Guia de bares: es el encargado de almacenar los bares. A partir de este objeto es que el buscador puede empezar a buscar bares por distancias.

Objeto GPS: Encargado de calcular las posiciones de los bares respecto a su direccion. Conoce ademas la posicion actual del usuario y es capaz de medir distancias.

objeto Regulador de calificaciones:

Objeto XXX comentarios:

Objeto Mapa:

\subsection{Segundo Sprint}

Luego de la primer demo, en la que presentamos a nuestros inversores la funcionalidad del primer sprint, nos dimos cuenta que los usuarios 
no representaban parte valiosa de los funcionalidad. A partir de esto decidimos quitar los permisos de usuarios de nuestro problema. Asi consideramos un usuario generico con permiso de 
acceder a todos los metodos de la aplicion.

\subsection{A futuro}


