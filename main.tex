\documentclass[10pt,a4paper]{article}
\usepackage[utf8]{inputenc} % para poder usar tildes en archivos UTF-8
\usepackage[spanish]{babel} % para que comandos como \today den el resultado en castellano
\usepackage{a4wide} % márgenes un poco más anchos que lo usual
\usepackage[conEntregas]{caratula}
\usepackage{mathtools}
\usepackage{float}
\usepackage[pdftex]{graphicx}
\usepackage{caption}
\usepackage{subcaption}
%\usepackage{Sty/algorithm2e}
\usepackage[ruled,vlined]{algorithm2e}
%Esto de abajo es para encabezado y pie de pagina
\usepackage{lastpage}
\usepackage{fancyhdr}
\usepackage{amsfonts}
\usepackage{verbatim}

%Esto es para configurar márgenes
\textheight=25cm
\textwidth=18cm
\topmargin=-1.5cm 
\oddsidemargin=-1cm 

\usepackage{wrapfig}
\usepackage{multirow} 

\pagestyle{fancy}
%\fancyhf{} % clear all header and footer fields
%\fancyfoot[R]{\footnotesize Página \thepage\ de \lastpage\}

\cfoot{\thepage /\pageref{LastPage} }

\newcommand\BlockIf[1]{\KwSty{If} \\ #1 \\ \KwSty{End If}}
\newcommand\BlockElseIf[1]{\KwSty{Else If} \\ #1 \\ \KwSty{End Else If}}
\newcommand\BlockElse[1]{\KwSty{Else} \\ #1 \\ \KwSty{End Else}}

\begin{document}

\titulo{Trabajo Práctico}
\subtitulo{Primer sprint backlog}

\fecha{\today}

\materia{Ingeniería de Software 1}
\grupo{}

% Pongan cuantos integrantes quieran
\integrante{Abdala Leila}{950/12}{abdalaleila@gmail.com}
\integrante{Bernaus Andres}{699/10}{andres.bernaus@hotmail.com}
\integrante{Gonzalez Alejandro}{32/13}{gonzalezalejandro1592@gmail.com}
\integrante{Sabarros Ian}{661/11}{iansden@live.com}
\integrante{Torrico Mirko}{28/10}{mirko.torrico@gmail.com}



\maketitle

\newpage
\begin{comment}
%A continuacion detallamos las User Stories que definimos para este proyecto.
\begin{table}[htbp]
\begin{center}
\begin{tabular}{|l|p{1.7cm}|p{3cm}|p{1cm}|p{7cm} |}
\hline
User stories \\
\hline \hline 
 Como: & Quiero: & Para: & Stories points: & Criterios de aceptacion: \\
\hline \hline
Usuario No Logueado & Registrarme & Loguearme & 2 & El usuario debe ingresar su dirección de email, nombre de usuario y contrase\~na, luego
se ingresa la información en la base de datos.\\ \hline
Usuario No Logueado & Loguearme & Poder acceder a todas las funcionalidades y a mi historial en la aplicacion. & 1 & El usuario debe ingresar su nombre de usuario y contrase\~na, luego
se verifica la información en la base de datos.\\
\hline
Usuario Logueado & Buscar bar por distancia & Seleccionar un bar de acuerdo a mis preferencias & 2 & El usuario debe poder visualizar los distintos 
bares dentro de un radio de 400mts. Al seleccionar uno de los bares de la lista se debe mostrar informacion relevante del bar. El usuario luego podra 
volver al menu anterior o "confirmar" el bar y recibir un mapa con la ubicacion del bar, la ubicacion del usuario y el camino entre ellos.
\\ \hline

Usuario Logueado & Acceder al historial de busquedas & Poder volver a ver la informacion de bares visitados & 3 & El usuario debe poder visualizar todos los bares que alguna vez eligio.
El usuario debe poder seleccionar uno de estos bares y obtener informacion detallada del mismo (comentarios, calificaciones, informacion general, etc).
El usuario podra reportar un comentario hecho por otro usuario para ser revisado luego por otro moderador.\\
\hline

Usuario Logueado & Evaluar o comentar un bar & Dejar acentada una critica del mismo & 3 &
El usuario debe poder escribir un nuevo comentario.
El usuario debe poder dar una calificacion a cuaquiera de las categorias del Bar.
Si el usuario ya habia dado una calificacion en una categoria ya puntuada se realizara el cambio. Es decir, se eliminara la calificacion anterior y se la reemplazara por la nueva.
El usuario podra editar o eliminar un comentario que haya realizado sobre dicho bar.\\ \hline

Usuario Logueado & Seleccionar una o mas categorias existentes & Realizar una busqueda con prioridad sobre dichos parametros & 2 &
El usuario debe poder visualizar todas las opciones posibles a la hora de elegir una categoria.
El usuario debe poder elegir una categoria y luego la lista de resultados debe ordenarse conforme a dicha eleccion.\\ \hline

Usuario Logueado & Solicitar el registro de un nuevo bar en la base de datos & Compartir informacion sobre dicho bar & 2 &
La direccion enviada por el usuario debe ser valida (debe existir).
Una vez ingresada la direccion y el nombre del nuevo bar se debe crear una nueva entrada en el sistema para notificar el deseo de los usuarios de dar de alta el mismo.
\\ \hline

Moderador & Eliminar un bar de la base de datos & Atender solicitudes de baja por parte de usuarios. & 5 &
El moderador debe ingresar la direccion del local
Una vez ingresada se validara y obtendra la latitud/longitud de la misma.
El moderador obtendra el bar buscado y podra marcarlo como eliminado. Evitando que los usuarios puedan acceder a la informacion del mismo.

\\ \hline

\end{tabular}
\end{center}
\end{table}\newpage


\begin{table}[htbp][H]
\begin{center}
\begin{tabular}{|l|p{1.7cm}|p{3cm}|p{1cm}|p{8cm} |}
\hline
 Como: & Quiero: & Para: & Stories points: & Criterios de aceptacion: \\
\hline \hline


Moderador & Agrega un bar a la base de datos & Atender solicitudes de alta por parte de usuarios. & 2 &

El moderador debe ingresar la direccion del local
Una vez ingresada se validara y obtendra la latitud/longitud de la misma.
El moderador obtendra el bar buscado y podra crear una nueva entrada con toda la informacion obtenida (nombre, lat, long, calle y categorias necesarias, etc).

\\ \hline


Moderador & Acceder y poder eliminar los comentarios repostados & Atender quejas por parte de usuarios. & 2 &
El moderador debe poder observar todos los comentarios que hayan sido reportados. Estos deben estar en orden descendente segun la cantidad de "solicitudes".
Aquellos comentarios con menos de $\lambda$ solicitudes no seran mostrados.
El moderador debe poder eliminar el comentario.
El moderado de debe poder marcar un comentario como verificado.
El moderador debe poder enviar un ``aviso''(warning) al usuario, en caso de tener $\mu$ cantidad el usuario sera suspendido.\\
\hline

Moderador & Otorgar permiso de moderador a un usuario & Ayudar a administrar la aplicacion. & 2 &
El moderador debe poder ingresar un usuario y, en caso de existir en el sistema, debera aparecer como resultado.
Una vez encontrado podra marcarlo como moderador.
El usuario moderado podra acceder a todas las funcionalidades correspondientes.
\\
\hline

Moderador & Acceder a las solicitudes de alta de bares por parte de los usuarios & Determinar que bares registrar & 3 &
El moderador debera poder visualizar una lista de direcciones y nombres ordenadas por cantidad de solicitudes.
El moderador podra dar de alta el bar o rechazar las solicitudes generadas (en caso de que el bar haya sido dado de baja).

\\
\hline

Moderador & Agregar nuevas caracteristicas para evaluar los bares & Brindar mayor informacion a los usuarios & 2 &
El moderador debe poder ingresar la categoria deseada.
Una vez ingresada la categoria cualquier bar del sistema debera tener la nueva categoria como parte de su informacion.

\\ \hline

\end{tabular}
\end{center}
\end{table}
\newpage


\begin{table}[htbp]
\begin{center}
\begin{tabular}{|l|p{8cm}|p{1cm}|p{1cm}|}
\hline
Prime Sprint:\\ \hline \hline

User Storie & Tareas: & Horas: & Total: \\
\hline \hline

\multirow{3}{2cm}{Buscar bar por distacia} & Crear entorno de programacion & 2 & \\ \cline{2-4}
& Medir distacias a nuestra direccion & 3 & \\ \cline{2-4}
& Crear interfaz grafica & 3 & 8\\ \cline{1-4}
\hline \hline
\multirow{3}{2cm}{Agregar bar} & Crear entorno de programacion & 2 & \\ \cline{2-4}
& Ubicar bar en el mapa & 2 & \\ \cline{2-4}
& Crear interfaz grafica & 3 & 7\\ \cline{1-4}

\hline \hline
\multirow{3}{2cm}{Registrarme} & Crear entorno de programacion & 2 & \\ \cline{2-4}
& Verificar datos y agregarlos a la base de datos & 5 & \\ \cline{2-4}
& Crear interfaz grafica & 2 & 9\\ \cline{1-4}
\hline \hline
\multirow{3}{2cm}{Loguearme} & Crear entorno de programacion & 2 & \\ \cline{2-4}
& Verificar datos & 1 & \\ \cline{2-4}
& Crear interfaz grafica & 1 & 4\\ \cline{1-4}
\hline \hline
\multirow{3}{2cm}{Eliminar bar} & Crear entorno de programacion & 2 & \\ \cline{2-4}
& Verificar datos del bar& 1 & \\ \cline{2-4}
& Verificar datos del usuario& 1 & \\ \cline{2-4}
& Eliminar datos de la base de datos& 5 & \\ \cline{2-4}
& Crear interfaz grafica & 3 & 13\\ \cline{1-4}

\end{tabular}
\end{center}
\end{table}
\end{comment}

\section{Introducción}
Dada la pachorra de cierto grupo de estudiantes para juntarse a estudiar en una biblioteca, nos encontramos con el problema de localizar bares con ciertas caracteristicas del 
agrado de los pachorrientos estudiantes. Estos mismos proveyeron una muy breve y ambigua descripcion de su problema, el cual pretendemos resolver mediante el desarrrollo de una 
aplicacion web llamada Wi-FindBar (porque algo solo es cheto si esta en ingles). 

El presente informe muestra el diseno de la aplicacion, asi como el desarrollo de la misma hasta el dia de la fecha. Para el diseno se uso programacion orientada a objetos, 
principalmente las tecnicas de doble dispatch y polimorfismo. Para el desarrollo usamos la porqueria (...) de Scrum, ya que nos permite empezar a trabajar de manera incremental, sin tener
pleno conocimiento del cominio del problema.

\section{Backlog}
A continuacion presentamos el conjunto de User Stories que conformaran nuestra aplicacion, con las decisiones que acompana a cada una.

En principio identificamos los roles involucrados.
\begin{itemize}
 \item Buscador de bares: Este rol se encarga de realizar todas las busquedas, ya sea por distancia, por categoria o buscar la menor ruta a determinado bar.
 \item Editor de guias: Es el encargado de la guia de bares. Sus responsabilidades se limitan a agragar y eliminar bares en la guia.
 \item Comentador de bares: Este rol administra los comentarios.
 \item Calificador de bares: Este rol es el encargado de conocer y manejar las categorias de clasificacion de los bares, asi como las calificaciones que estos reciben de los usuarios.
\end{itemize}


\subsection{User Stories}
Las blahblahblah (...) user stories representan los casos blahblahblah

\begin{enumerate}
 \item US
 
Queremos poder filtrar los bares que esten a 400 mts de mi posicion. Esto es para determinar caules son los bares que estan a una distancia razonable para llegar caminando. Ya que nuestra
aplicacion consiste basicamente en un buscador de bares, consideramos que esto forma parte de la funcionaldad indispensable esperada por nuestros inversores %putos
por lo que tiene un gran Bussines Value.
\paragraph{Como:} Buscador.
\paragraph{Quiero:} Filtrar bares por distancia.
\paragraph{Para:} Determinar, de acuerdo a mis requerimientos, cuales son los bares plausibles que puedo visitar.
\paragraph{Story Point:} 3
\paragraph{Bussines Value:} 10


 \item US
 
Nuestros usuarios quieren no solo poder ver la distancia, sino que les interesa que los bares devueltos cumplan con ciertas caracteristicas particulares. Como para cada usuario estos
requerimientos son diferentes, esta us contempla poder evaluar varias categorias diferentes. Ademas, como en el caso anterior, esto es parte indispensable de la funcionalidades
 por lo que el Bussines Value es equivalente al caso anterior.
\paragraph{Como:} Buscador
\paragraph{Quiero:} filtrar por categoria
\paragraph{Para:}
\paragraph{Story Point:} 3
\paragraph{Bussines Value:} 10

 \item US
 
 
\paragraph{Como:} Buscador
\paragraph{Quiero:} ver camino minimo
\paragraph{Para:}
\paragraph{Story Point:} 3
\paragraph{Bussines Value:} 8 

 \item US
\paragraph{Como:} Editor
\paragraph{Quiero:} agregar bar
\paragraph{Para:}
\paragraph{Story Point:} 1
\paragraph{Bussines Value:} 8


 \item US
\paragraph{Como:} Editor
\paragraph{Quiero:} eliminar bar
\paragraph{Para:}
\paragraph{Story Point:}1 
\paragraph{Bussines Value:}5



 \item US
\paragraph{Como:} Comentador
\paragraph{Quiero:} hacer comentario
\paragraph{Para:}
\paragraph{Story Point:}3
\paragraph{Bussines Value:}5


 \item US
\paragraph{Como:} Comentador
\paragraph{Quiero:}ver comentarios de un bar 
\paragraph{Para:}
\paragraph{Story Point:}2
\paragraph{Bussines Value:} 5


 \item US
\paragraph{Como:} calificador (votador)
\paragraph{Quiero:} votar bar por categoria x
\paragraph{Para:}
\paragraph{Story Point:}1
\paragraph{Bussines Value:} 8


 \item US
\paragraph{Como:} Calificador (votador)
\paragraph{Quiero:}ver calificaciones
\paragraph{Para:}
\paragraph{Story Point:}2
\paragraph{Bussines Value:}5

\end{enumerate}




\section{Burndown Charts}
 

\section{Product Increment}
 

\section{Retrospectiva}
\section{Retrospectiva}
En esta seccion chamuyaremos sobre los problemas hallados y demas bobadas (...) ocurridas durante los sprint.

\subsection{Primer diseño}

En principio planteamos un diseño sobre el cual iteramos en cada sprint. A continuacion los describiremos brevemente para darl al lector una idea de la evolucion del diseño, sin embargo
no haremos demaciado incapie en esto pues no es el diseño final sobre el que trabajamos.

En nuestro diseño consideramos como centrales los siguientes puntos. 

Metodo buscar bares: La idea de este metodo era tener un mensaje polimorfico al que pudiesemos pasarle cualquier tipo de
requerimientos (los cuales probablemente deberian haber sido implementador mediante cloures) y devolviera la lista de bares que los cumpliesen. Para esto El objeto encargado 
de buscar los bares deberia conocer la forma de almacenarlos.

Roles: Nuestros principales roles se basaron en principio en los distintos tipos de permisos que tendrian los usuarios, siendo ellos No registrado, Registrado y Moderador. A cada 
uno de estos les correspondian diferentes funcionalidades que podian o no pertenecer a uno de los otros usarios. Por ejemplo, un usuario no registrado puede buscar al igual que uno registrado
o que un moderados, sin embargo, el moderador podia eliminar bares, cosa que un usuario registrado no podria.

Objeto bases de datos: Consideramos tres bases de datos diferentes, una para regitras usuarios, una para almacenar la informacion de los bares y otra par aalmacenar los comentarios y calificaciones
realizadas por los usuarios. Esta tres bases de datos estaban fuertemente acopladas entre si, ya que, por ejemplo, la base de datos de votaciones conocia a los bares y a los usuarios.

Objeto bar: El objeto bar contenia toda la informacion asociada al mismo. Es decir, nombre, direccion, pocision, sus calificaciones, comentarios y una brebe descripcion del mismo. Es 
decir, que el objeto bar contenia toda la informacion necesaria para responder por si mismo la toda la funcionalidad pedida. 

Metodos de administracion de bares: Estos metodos se referian a la creacion, eliminacion y modificacion de bares, teniendo acceso a la base de datos de bares y de votaciones.


Metodos de administracion de usuarios: Estos metodos se referian a la creacion, eliminacion y modificacion de usuarios, teniendo acceso a la base de datos de usuarios y de votaciones.


\subsection{Primer Sprint}

Durante el primer sprint detectamos problemas con el diseno de los usuarios. Al querer modificar el status de un usuario debiamos destruir el usuario y crearlo de nuevo, asignandole su nuevo
permiso. Esto fue debido a que habiamos definido una clase usuario abstracta de la cual se heredaban los usuarios logueado y no logueado. Para evitar esto, definimos un ingresante a la aplicacion sin ningun permiso, que solo podia buscar. Al loquearse este era destruido y se creaba
un nuevo objeto que representaba un usuario ya logueado.

Los nuevos roles que consideramos fueron seleccionados de acuerdo a la funcionalidad que proveia cada uno. Asi indentificamos a los roles Buscador de bares, Editor de guias, Comentador
de bares, Calificador de bares, descriptos en la seccion 2. A partir de estos surgieron nuevos objetos y metodos, y se modificaron los ya existentes.

Objeto Bar: Dada el alto acloplamiento que tenia este objeto, lo redisenamos intentando capturar solo su esencia. Esto resulto en que el objeto bar solo conoce su nombre y direccion. 
Los demas datos que almacenaba delegados a nuevos objetos.

Objeto Guia de bares: es el encargado de almacenar los bares. A partir de este objeto es que el buscador puede empezar a buscar bares por distancias.

Objeto GPS: Encargado de calcular las posiciones de los bares respecto a su direccion. Conoce ademas la posicion actual del usuario y es capaz de medir distancias.

objeto Regulador de calificaciones:

Objeto XXX comentarios:

Objeto Mapa:

\subsection{Segundo Sprint}

Luego de la primer demo, en la que presentamos a nuestros inversores la funcionalidad del primer sprint, nos dimos cuenta que los usuarios 
no representaban parte valiosa de los funcionalidad. A partir de esto decidimos quitar los permisos de usuarios de nuestro problema. Asi consideramos un usuario generico con permiso de 
acceder a todos los metodos de la aplicion.

\subsection{A futuro}




\section{Diseno}
\subsection{Diseño Orientado a Objetos y justificación}

Para el diseño vamos a usar los siguientes escenarios posibles en el marco del uso de la aplicación:

\begin{enumerate}
\item Fernil fue a un bar llamado Los Trillizos, ubicado en microcentro. El ya conocía el bar por lo cual no lo busco previamente en la app. Después de pasar un buen rato decide realizar un comentario sobre el bar y dada unas carácteristicas dadas decide votar por las que le interesan, en particular atención de las meseras.

\item Javier sale con un grupo de amigos,se encuentran cercanos a una avenida importante y cerca de un bar que no conoce. Para no pasar una mala noche, decide buscar en la app los bares cercanos a su posición. Pero entre los resultados no aparece el bar que esta viendo con sus propios ojos. Por otro lado los otros bares que le aparecen son bares que no le quedan lo suficientemente cerca. Dado esto decide entrar al bar. Finalmente, como pasó una muy buena noche y recordó que ese bar no estaba en la app, decide agregarlo a la app, Para ellos completa un par de datos como: el nombre del bar la dirección y una pequeña reseña sobre el mismo. 

\item Leila se encuentra en su casa, en espera de un grupo de amigos para terminar con un tp. Tiene dos opciones quedarse en casa o ir a un lugar cercano. Como no le gusta mucho la idea de que se queden en casa, decide buscar algo cerca. Usa la app con la intencción de buscar un bar cercano de su casa y que además tengan buenas puntuaciones con respecto a wifi y enchufez ya que van a usar de manera exaustiva sus respectivas laptos.

\item Dos amigos se encuentran en las cercanias de Plaza Italia, ambos vienen de lugares totalmente distintos pero deciden verse ahí porque cada uno tiene un colectivo de distancia. Una vez que se encuetran, no tienen ni idea a donde ir. Por medio de la app uno de ellos decide buscar algo cercano. Luego de ver un conjunto de bares cercanos se deciden por uno. Dado a que ninguno es de la zona prefieren que la app les muestre por pantalla una forma de dirigirse al bar que eligieron ir.


\end{enumerate}


\section{Conclusión}
 \section{Conclusión}



\end{document}
