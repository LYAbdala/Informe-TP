\documentclass[10pt,a4paper]{article}
\usepackage[utf8]{inputenc} % para poder usar tildes en archivos UTF-8
\usepackage[spanish]{babel} % para que comandos como \today den el resultado en castellano
\usepackage{a4wide} % márgenes un poco más anchos que lo usual
\usepackage[conEntregas]{caratula}
\usepackage{mathtools}
\usepackage{float}
\usepackage[pdftex]{graphicx}
\usepackage{caption}
\usepackage{subcaption}
%\usepackage{Sty/algorithm2e}
\usepackage[ruled,vlined]{algorithm2e}
%Esto de abajo es para encabezado y pie de pagina
\usepackage{lastpage}
\usepackage{fancyhdr}
\usepackage{amsfonts}
\usepackage{verbatim}

%Esto es para configurar márgenes
\textheight=25cm
\textwidth=18cm
\topmargin=-1.5cm 
\oddsidemargin=-1cm 

\usepackage{wrapfig}
\usepackage{multirow} 

\pagestyle{fancy}
%\fancyhf{} % clear all header and footer fields
%\fancyfoot[R]{\footnotesize Página \thepage\ de \lastpage\}

\cfoot{\thepage /\pageref{LastPage} }

\newcommand\BlockIf[1]{\KwSty{If} \\ #1 \\ \KwSty{End If}}
\newcommand\BlockElseIf[1]{\KwSty{Else If} \\ #1 \\ \KwSty{End Else If}}
\newcommand\BlockElse[1]{\KwSty{Else} \\ #1 \\ \KwSty{End Else}}

\begin{document}

\titulo{Trabajo Práctico}
\subtitulo{Primer sprint backlog}

\fecha{\today}

\materia{Ingeniería de Software 1}
\grupo{}

% Pongan cuantos integrantes quieran
\integrante{Abdala Leila}{950/12}{abdalaleila@gmail.com}
\integrante{Bernaus Andres}{699/10}{andres.bernaus@hotmail.com}
\integrante{Gonzalez Alejandro}{32/13}{gonzalezalejandro1592@gmail.com}
\integrante{Sabarros Ian}{661/11}{iansden@live.com}
\integrante{Torrico Mirko}{28/10}{mirko.torrico@gmail.com}



\maketitle

\newpage
\begin{comment}
%A continuacion detallamos las User Stories que definimos para este proyecto.
\begin{table}[htbp]
\begin{center}
\begin{tabular}{|l|p{1.7cm}|p{3cm}|p{1cm}|p{7cm} |}
\hline
User stories \\
\hline \hline 
 Como: & Quiero: & Para: & Stories points: & Criterios de aceptacion: \\
\hline \hline
Usuario No Logueado & Registrarme & Loguearme & 2 & El usuario debe ingresar su dirección de email, nombre de usuario y contrase\~na, luego
se ingresa la información en la base de datos.\\ \hline
Usuario No Logueado & Loguearme & Poder acceder a todas las funcionalidades y a mi historial en la aplicacion. & 1 & El usuario debe ingresar su nombre de usuario y contrase\~na, luego
se verifica la información en la base de datos.\\
\hline
Usuario Logueado & Buscar bar por distancia & Seleccionar un bar de acuerdo a mis preferencias & 2 & El usuario debe poder visualizar los distintos 
bares dentro de un radio de 400mts. Al seleccionar uno de los bares de la lista se debe mostrar informacion relevante del bar. El usuario luego podra 
volver al menu anterior o "confirmar" el bar y recibir un mapa con la ubicacion del bar, la ubicacion del usuario y el camino entre ellos.
\\ \hline

Usuario Logueado & Acceder al historial de busquedas & Poder volver a ver la informacion de bares visitados & 3 & El usuario debe poder visualizar todos los bares que alguna vez eligio.
El usuario debe poder seleccionar uno de estos bares y obtener informacion detallada del mismo (comentarios, calificaciones, informacion general, etc).
El usuario podra reportar un comentario hecho por otro usuario para ser revisado luego por otro moderador.\\
\hline

Usuario Logueado & Evaluar o comentar un bar & Dejar acentada una critica del mismo & 3 &
El usuario debe poder escribir un nuevo comentario.
El usuario debe poder dar una calificacion a cuaquiera de las categorias del Bar.
Si el usuario ya habia dado una calificacion en una categoria ya puntuada se realizara el cambio. Es decir, se eliminara la calificacion anterior y se la reemplazara por la nueva.
El usuario podra editar o eliminar un comentario que haya realizado sobre dicho bar.\\ \hline

Usuario Logueado & Seleccionar una o mas categorias existentes & Realizar una busqueda con prioridad sobre dichos parametros & 2 &
El usuario debe poder visualizar todas las opciones posibles a la hora de elegir una categoria.
El usuario debe poder elegir una categoria y luego la lista de resultados debe ordenarse conforme a dicha eleccion.\\ \hline

Usuario Logueado & Solicitar el registro de un nuevo bar en la base de datos & Compartir informacion sobre dicho bar & 2 &
La direccion enviada por el usuario debe ser valida (debe existir).
Una vez ingresada la direccion y el nombre del nuevo bar se debe crear una nueva entrada en el sistema para notificar el deseo de los usuarios de dar de alta el mismo.
\\ \hline

Moderador & Eliminar un bar de la base de datos & Atender solicitudes de baja por parte de usuarios. & 5 &
El moderador debe ingresar la direccion del local
Una vez ingresada se validara y obtendra la latitud/longitud de la misma.
El moderador obtendra el bar buscado y podra marcarlo como eliminado. Evitando que los usuarios puedan acceder a la informacion del mismo.

\\ \hline

\end{tabular}
\end{center}
\end{table}\newpage


\begin{table}[htbp][H]
\begin{center}
\begin{tabular}{|l|p{1.7cm}|p{3cm}|p{1cm}|p{8cm} |}
\hline
 Como: & Quiero: & Para: & Stories points: & Criterios de aceptacion: \\
\hline \hline


Moderador & Agrega un bar a la base de datos & Atender solicitudes de alta por parte de usuarios. & 2 &

El moderador debe ingresar la direccion del local
Una vez ingresada se validara y obtendra la latitud/longitud de la misma.
El moderador obtendra el bar buscado y podra crear una nueva entrada con toda la informacion obtenida (nombre, lat, long, calle y categorias necesarias, etc).

\\ \hline


Moderador & Acceder y poder eliminar los comentarios repostados & Atender quejas por parte de usuarios. & 2 &
El moderador debe poder observar todos los comentarios que hayan sido reportados. Estos deben estar en orden descendente segun la cantidad de "solicitudes".
Aquellos comentarios con menos de $\lambda$ solicitudes no seran mostrados.
El moderador debe poder eliminar el comentario.
El moderado de debe poder marcar un comentario como verificado.
El moderador debe poder enviar un ``aviso''(warning) al usuario, en caso de tener $\mu$ cantidad el usuario sera suspendido.\\
\hline

Moderador & Otorgar permiso de moderador a un usuario & Ayudar a administrar la aplicacion. & 2 &
El moderador debe poder ingresar un usuario y, en caso de existir en el sistema, debera aparecer como resultado.
Una vez encontrado podra marcarlo como moderador.
El usuario moderado podra acceder a todas las funcionalidades correspondientes.
\\
\hline

Moderador & Acceder a las solicitudes de alta de bares por parte de los usuarios & Determinar que bares registrar & 3 &
El moderador debera poder visualizar una lista de direcciones y nombres ordenadas por cantidad de solicitudes.
El moderador podra dar de alta el bar o rechazar las solicitudes generadas (en caso de que el bar haya sido dado de baja).

\\
\hline

Moderador & Agregar nuevas caracteristicas para evaluar los bares & Brindar mayor informacion a los usuarios & 2 &
El moderador debe poder ingresar la categoria deseada.
Una vez ingresada la categoria cualquier bar del sistema debera tener la nueva categoria como parte de su informacion.

\\ \hline

\end{tabular}
\end{center}
\end{table}
\newpage


\begin{table}[htbp]
\begin{center}
\begin{tabular}{|l|p{8cm}|p{1cm}|p{1cm}|}
\hline
Prime Sprint:\\ \hline \hline

User Storie & Tareas: & Horas: & Total: \\
\hline \hline

\multirow{3}{2cm}{Buscar bar por distacia} & Crear entorno de programacion & 2 & \\ \cline{2-4}
& Medir distacias a nuestra direccion & 3 & \\ \cline{2-4}
& Crear interfaz grafica & 3 & 8\\ \cline{1-4}
\hline \hline
\multirow{3}{2cm}{Agregar bar} & Crear entorno de programacion & 2 & \\ \cline{2-4}
& Ubicar bar en el mapa & 2 & \\ \cline{2-4}
& Crear interfaz grafica & 3 & 7\\ \cline{1-4}

\hline \hline
\multirow{3}{2cm}{Registrarme} & Crear entorno de programacion & 2 & \\ \cline{2-4}
& Verificar datos y agregarlos a la base de datos & 5 & \\ \cline{2-4}
& Crear interfaz grafica & 2 & 9\\ \cline{1-4}
\hline \hline
\multirow{3}{2cm}{Loguearme} & Crear entorno de programacion & 2 & \\ \cline{2-4}
& Verificar datos & 1 & \\ \cline{2-4}
& Crear interfaz grafica & 1 & 4\\ \cline{1-4}
\hline \hline
\multirow{3}{2cm}{Eliminar bar} & Crear entorno de programacion & 2 & \\ \cline{2-4}
& Verificar datos del bar& 1 & \\ \cline{2-4}
& Verificar datos del usuario& 1 & \\ \cline{2-4}
& Eliminar datos de la base de datos& 5 & \\ \cline{2-4}
& Crear interfaz grafica & 3 & 13\\ \cline{1-4}

\end{tabular}
\end{center}
\end{table}
\end{comment}

\section{Introduccion}

Dada la pachorra de cierto grupo de estudiantes para juntarse a estudiar en una biblioteca, nos encontramos con el problema de localizar bares con ciertas caracteristicas del 
agrado de los pachorrientos estudiantes. Estos mismos proveyeron una muy breve y ambigua descripcion de su problema, el cual pretendemos resolver mediante el desarrrollo de una 
aplicacion web llamada Wi-FindBar (porque algo solo es cheto si esta en ingles). 

El presente informe muestra el diseno de la aplicacion, asi como el desarrollo de la misma hasta el dia de la fecha. Para el diseno se uso programacion orientada a objetos, 
principalmente las tecnicas de doble dispatch y polimorfismo. Para el desarrollo usamos la porqueria (...) de Scrum, ya que nos permite empezar a trabajar de manera incremental, sin tener
pleno conocimiento del cominio del problema.

\section{User Stories}

A continuacion presentamos el conjunto de User Stories que conformaran nuestra aplicacion, con las decisiones que acompana a cada una.

En principio identificamos los roles involucrados.
\begin{itemize}
 \item Buscador: Este rol se encarga de realizar todas las busquedas, ya sea por distancia, por categoria o buscar la menor ruta a determinado bar.
 \item Editor: Es el 
 \item Comentador:
 \item Votador:
 
 
\end{itemize}


\section{Burndown Charts}

\section{Product Increment}

\section{Retrospectiva}

\section{Diseno}




\end{document}
